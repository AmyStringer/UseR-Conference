% Copyright 2004 by Till Tantau <tantau@users.sourceforge.net>.


\documentclass{beamer}
\usepackage{amsmath, amsthm, amssymb}
\usepackage{graphicx}
\usepackage{float}

\usetheme{CambridgeUS}         



\title{Automated Summarisation of Big Data using Rmarkdown}

% A subtitle is optional and this may be deleted
\subtitle{Using data from the Catlin Seaview Survey - a global coral reef monitoring effort}

\author{Amy Stringer}
% - Give the names in the same order as the appear in the paper.
% - Use the \inst{?} command only if the authors have different
%   affiliation.

\institute[Global Change Institute] % (optional, but mostly needed)
{
  \inst{1}%
  University of Queensland
}
% - Use the \inst command only if there are several affiliations.
% - Keep it simple, no one is interested in your street address.

\date{UseR! July 2018}
% - Either use conference name or its abbreviation.
% - Not really informative to the audience, more for people (including
%   yourself) who are reading the slides online


% If you have a file called "university-logo-filename.xxx", where xxx
% is a graphic format that can be processed by latex or pdflatex,
% resp., then you can add a logo as follows:

\pgfdeclareimage[height=0.5cm]{university-logo}{logos}
\logo{\pgfuseimage{university-logo}}

% Delete this, if you do not want the table of contents to pop up at
% the beginning of each subsection:
\AtBeginSubsection[]
{
  \begin{frame}<beamer>{Outline}
    \tableofcontents[currentsection,currentsubsection]
  \end{frame}
}

% Let's get started
\begin{document}
    \section{Introduction}
        \begin{frame}
          \titlepage
        \end{frame}
        
        \begin{frame}{Outline}
          \tableofcontents
          % You might wish to add the option [pausesections]
        \end{frame}

% Section and subsections will appear in the presentation overview
% and table of contents.
        \begin{frame}{Key Exchange}
        

            
        \end{frame}

    \subsection{}




    \subsection{}

% You can reveal the parts of a slide one at a time
% with the \pause command:

%%%% SLIDE 6 %%%% 

        \begin{frame}{The Age of Quantum Computers}
          
        %   \begin{itemize}
        %   \item {
        %     First item.
        %     \pause % The slide will pause after showing the first item
        %   }
        %   \item {   
        %     Second item.
        %   }
        %   % You can also specify when the content should appear
        %   % by using <n->:
        %   \item<3-> {
        %     Third item.
        %   }
        %   \item<4-> {
        %     Fourth item.
        %   }
        %   % or you can use the \uncover command to reveal general
        %   % content (not just \items):
        %   \item<5-> {
        %     Fifth item. \uncover<6->{Extra text in the fifth item.}
        %   }
        %   \end{itemize}
        \end{frame}
        

        
\section{Learning with Errors (LWE)}

    \subsection{The Learning with Errors Problem (LWE)}

%%%% SLIDE 8 %%%% 

        

    \subsection{Key Exchange with LWE}
    
%%% SLIDE 11 %%%% 

        \begin{frame}{Key Exchange with LWE}
        
            
        \end{frame}

        % \begin{frame}{Blocks}
        %     \begin{block}{Block Title}
        %         You can also highlight sections of your presentation in a block, with it's own title
        %     \end{block}
        %     \begin{theorem}
        %         There are separate environments for theorems, examples, definitions and proofs.
        %     \end{theorem}
        %     \begin{example}
        %         Here is an example of an example block.
        %     \end{example}
        % \end{frame}


\section{Chinese Remainder with Errors} 

    \subsection{Chinese Remainder Theorem} 
    
%%%% SLIDE 13 %%%% 

        
        

    \subsection{Candidate for Key Exchange}
        \subsubsection{Some Constraints}

%%%% SLIDE 15 %%%% 

        \begin{frame}{Constraints}
                    
            
        \end{frame}
     
%%%% SLIDE 16 %%%% 

        \begin{frame}{Constraints cont.}
        
            

        \end{frame}
        
        \subsubsection{Progress on Key Exchange}
        
%%%% SLIDE 17 & 18 %%%% 

        \begin{frame}{3 Attempts so far}{First and Second Attempt}
            
        \end{frame}

%%%% SLIDE 18 %%%% 
        
        \begin{frame}{3 Attempts so far}{Third Attempt - In progress}
            
            
            
        \end{frame}
% Placing a * after \section means it will not show in the
% outline or table of contents.
\section*{Summary}

%%%% SLIDE 19 %%%% 

\begin{frame}{Summary}
  


\end{frame}



% All of the following is optional and typically not needed. 
\appendix
\section<presentation>*{\appendixname}
\subsection<presentation>*{For Further Reading}

\begin{frame}[allowframebreaks]
  \frametitle<presentation>{For Further Reading}
    
  \begin{thebibliography}{10}
    
  \beamertemplatebookbibitems
  % Start with overview books.
  
  
    

  \end{thebibliography}
\end{frame}

\end{document}


